\documentclass[letterpaper,10pt]{article}

\usepackage{latexsym}
\usepackage[empty]{fullpage}
\usepackage{titlesec}
\usepackage{marvosym}
\usepackage[usenames,dvipsnames]{color}
\usepackage{verbatim}
\usepackage{enumitem}
\usepackage[hidelinks]{hyperref}
\usepackage{fancyhdr}
\usepackage[english]{babel}
\usepackage{tabularx}
\usepackage{xcolor}
\usepackage{fontawesome5}
\usepackage{graphicx}

\input{glyphtounicode}

% Anpassung der Ränder
\addtolength{\oddsidemargin}{-0.5in}
\addtolength{\evensidemargin}{-0.5in}
\addtolength{\textwidth}{1in}
\addtolength{\topmargin}{-0.5in}
\addtolength{\textheight}{1in}

\urlstyle{same}
\raggedbottom
\raggedright
\setlength{\tabcolsep}{0in}

% Abschnittsformatierung
\titleformat{\section}{
  \vspace{-5pt}\scshape\raggedright\large
}{}{0em}{}[\color{black}\titlerule \vspace{-5pt}]
\titleformat{\subsection}{
  \vspace{-4pt}\scshape\raggedright\large
}{\hspace{-.15in}}{0em}{}[\color{black}\vspace{-8pt}]

\pdfgentounicode=1

% Benutzerdefinierte Befehle
\newcommand{\resumeItem}[1]{\item\small{#1 \vspace{-2pt}}}
\newcommand{\resumeSubheading}[4]{
  \vspace{-2pt}\item
    \begin{tabular*}{0.97\textwidth}[t]{l@{\extracolsep{\fill}}r}
      \textbf{#1} & #2 \\
      \textit{\small#3} & \textit{\small #4} \\
    \end{tabular*}\vspace{-7pt}
}
\newcommand{\resumeProjectHeading}[2]{
    \item
    \begin{tabular*}{0.97\textwidth}{l@{\extracolsep{\fill}}r}
      \small#1 & #2 \\
    \end{tabular*}\vspace{-7pt}
}
\newcommand{\resumeSubHeadingListStart}{\begin{itemize}[leftmargin=0.15in, label={}]}
\newcommand{\resumeSubHeadingListEnd}{\end{itemize}}
\newcommand{\resumeItemListStart}{\begin{itemize}}
\newcommand{\resumeItemListEnd}{\end{itemize}\vspace{-5pt}}

\renewcommand\labelitemii{$\vcenter{\hbox{\tiny$\bullet$}}$}

% Dokument
\begin{document}

% Kopfzeile
\begin{center}
    \textbf{\Huge \scshape Viswa Gandamalla} \\ \vspace{8pt}
    \small
    \faIcon{github} \href{https://github.com/viswa314}{github.com/viswa314} $|$
    \faIcon{linkedin} \href{https://de.linkedin.com/in/viswa-gandamalla-2064101b4/de}{linkedin.com/in/viswa-gandamalla-2064101b4} $|$
    \faIcon{envelope} \href{mailto:viswagandamalla@gmail.com}{viswagandamalla@gmail.com} $|$
    +4915758074771 $|$ Sebastian Straße 7, Ingolstadt- 85049
\end{center}

% Ausbildung
\section{Ausbildung}
  \resumeSubHeadingListStart
    \resumeSubheading
      {Technische Hochschule Ingolstadt}{Okt 2021 -- heute}
      {Master in International Automotive Engineering}{Ingolstadt, Deutschland}
      \resumeItemListStart
        \resumeItem{\textbf{Wichtige Kurse:} Integrierte Sicherheits- und Assistenzsysteme, Sensortechnologie und Signalverarbeitung, Automobil-Radarsysteme, Test- und Simulationsmethoden für Fahrzeugsicherheitssysteme und Fahrzeugdynamik}
        \resumeItem{\textbf{Auszeichnungen:} Deutschlandstipendium}
      \resumeItemListEnd

    \resumeSubheading
      {Vardhaman College of Engineering}{Mai 2017 -- Sep 2020}
      {Bachelor in Maschinenbau}{Hyderabad, Indien}
      \resumeItemListStart
        \resumeItem{\textbf{Auszeichnungen:} Akademischer Exzellenzpreis}
      \resumeItemListEnd
  \resumeSubHeadingListEnd

% Erfahrung
\section{Erfahrung}
  \resumeSubHeadingListStart
    \resumeSubheading
      {Akkodis}{Jan 2024 -- Jun 2024}
      {Masterarbeit über Objekterkennung für TurtleBot3}{Ingolstadt, Deutschland}
      \resumeItemListStart
        \resumeItem{Durchführung einer Studie zur Objekterkennung für TurtleBot3, Vergleich traditioneller Methoden (HOG+SVM, Haar-Cascades) und moderner Techniken (YOLOv8, RT-DETR).}
        \resumeItem{Implementierung und Bewertung von HOG+SVM und Haar-Cascades, wobei HOG+SVM eine mAP von 41,97\% erreichte, was die Grenzen traditioneller Methoden demonstrierte.}
        \resumeItem{Verbesserung der Echtzeiterkennung mit YOLOv8 (88,7\% mAP) und RT-DETR (81,3\% mAP), was die Robustheit unter variierenden Bedingungen erhöhte.}
        \resumeItem{Zusammenarbeit mit einem interdisziplinären Team zur Integration von Objekterkennungslösungen in breitere Projektziele, Präsentation der Ergebnisse in wöchentlichen Meetings.}
      \resumeItemListEnd
      \hspace*{\labelsep}\textbf{Technologie-Stack:} Python, ROS2, OpenCV, scikit-learn, PyTorch, ultralytics, Confluence, Jira, Git

    \resumeSubheading
      {Akkodis}{Dez 2023 -- Jan 2024}
      {Werkstudent}{Ingolstadt, Deutschland}
      \resumeItemListStart
        \resumeItem{Praktische Erfahrung mit Turtlebot3 unter Verwendung von ROS2 gesammelt.}
        \resumeItem{Nutzung des Atlassian-Toolchains (Jira, Confluence und Bitbucket) zur Verwaltung und Dokumentation des Projektfortschritts.}
        \resumeItem{Zusammenarbeit mit Teammitgliedern zur Fehlersuche und Optimierung der Verwendung der Realsense D405-Tiefenkamera.}
        \resumeItem{Kommunikation komplexer technischer Konzepte an nicht-technische Stakeholder zur Sicherstellung von Ausrichtung und Verständnis im Team.}
      \resumeItemListEnd
      \hspace*{\labelsep}\textbf{Technologie-Stack:} Python, ROS2, Confluence, Jira, Bitbucket, Git

    \resumeSubheading
      {Cognizant}{Jul 2023 -- Aug 2023}
      {Virtuelles Praktikum in Künstlicher Intelligenz von Forage}{}
      \resumeItemListStart
        \resumeItem{Datenanalyse zur Gewinnung von Erkenntnissen aus verschiedenen Datensätzen, Präsentation der Ergebnisse vor dem Senior Management.}
        \resumeItem{Effektive Kommunikation der Ergebnisse aus dem maschinellen Lernen an die Stakeholder, was zu umsetzbaren Geschäftsentscheidungen führte.}
        \resumeItem{Entwicklung von Fähigkeiten in Datenmanagement, KI und Analytik, Erkundung von Karrieremöglichkeiten innerhalb von Cognizant.}
        \resumeItem{Verbesserung der Datenverarbeitungseffizienz durch Implementierung optimierter Algorithmen.}
      \resumeItemListEnd
      \hspace*{\labelsep}\textbf{Technologie-Stack:} Python, scikit-learn

    \resumeSubheading
      {TechnoHacks EduTech}{Aug 2023 -- Sep 2023}
      {Virtuelles Praktikum im Bereich Maschinenlernen}{}
      \resumeItemListStart
        \resumeItem{Verwendung des Iris-Datensatzes zum Aufbau eines maschinellen Lernmodells zur Klassifikation von Irisblumen basierend auf ihren Sepal- und Petaldimensionen.}
        \resumeItem{Verwendung eines Datensatzes mit Transaktionsdaten zur Erkennung betrügerischer Transaktionen, dabei eine Genauigkeit von 85\% erreicht.}
        \resumeItem{Zusammenarbeit mit Kollegen zur Überprüfung und Verfeinerung von maschinellen Lernmodellen, was die Gesamtleistung um 10\% verbesserte.}
        \resumeItem{Präsentation der Projektergebnisse vor Mentoren und Kollegen, positives Feedback für Klarheit und Gründlichkeit erhalten.}
      \resumeItemListEnd
      \hspace*{\labelsep}\textbf{Technologie-Stack:} Python, ML-Regression und Klassifikation

    \resumeSubheading
      {CARISSMA}{Nov 2022 -- Apr 2023}
      {Forschungsstudentenassistent}{Ingolstadt, Deutschland}
      \resumeItemListStart
        \resumeItem{Implementierung des YOLO V4-Algorithmus zur Objekterkennung, Erhöhung der Erkennungsgenauigkeit um 12\%.}
        \resumeItem{Optimierung von Modellen zur Verbesserung der Genauigkeit und Echtzeitleistung, was zu einer Reduzierung der Verarbeitungszeit um 25\% führte.}
        \resumeItem{Datenerhebung und -analyse im Carla-Simulator durch Konfiguration der Umgebung, was zu verbesserten Trainingsdatensätzen für Modelle führte.}
        \resumeItem{Zusammenarbeit mit einem Forscherteam zur Präsentation der Ergebnisse auf einem departmental Seminar, erhielt Lob für detaillierte Analyse und Präsentationsfähigkeiten.}
      \resumeItemListEnd
      \hspace*{\labelsep}\textbf{Technologie-Stack:} Carla, Python, PyTorch
  \resumeSubHeadingListEnd

% Projekte
\section{Projekte}
  \resumeSubHeadingListStart
    \resumeProjectHeading
      {\textbf{Deep Learning für Verkehrszenario-Analysen}}{Okt 2021 -- Jan 2022}
      \resumeItemListStart
        \resumeItem{Analyse und Vorverarbeitung von Fleet-Daten aus der AWS-Cloud-Datenbank mit Numpy und Pandas.}
        \resumeItem{Präsentation einer Clustering-Lösung für Verkehrszenarien unter Verwendung von Unsupervised Learning mit CNN-Autoencodern in PyTorch und Validierung der Modellleistung.}
        \resumeItem{Zusammenarbeit mit einem Team zur Integration der Ergebnisse in breitere Verkehrsmanagementlösungen.}
      \resumeItemListEnd
      \hspace*{\labelsep}\textbf{Technologie-Stack:} Numpy, Pandas, PyTorch (konvolutionale neuronale Netze), Tensorboard

    \resumeProjectHeading
      {\textbf{Echtzeitfahrzeug- und Kennzeichenerkennung auf Edge-Geräten mit YOLO V4}}{Mär 2022 -- Jul 2022}
      \resumeItemListStart
        \resumeItem{Entwicklung eines hochgenauen Echtzeitfahrzeug- und Kennzeichenerkennungssystems mit YOLO V4 unter Verwendung von Transferlernen.}
        \resumeItem{Leistungsverbesserung durch Datensatzaugmentation, wodurch eine Erkennungsgenauigkeit von 95\% erreicht wurde.}
        \resumeItem{Erfolgreiche Bereitstellung auf NVIDIA Jetson Nano in einer eingebetteten Linux-Umgebung.}
        \resumeItem{Zusammenarbeit mit einem interdisziplinären Team zur Prüfung und Validierung des Systems, Präsentation der Ergebnisse an Stakeholder.}
      \resumeItemListEnd
      \hspace*{\labelsep}\textbf{Technologie-Stack:} Python, OpenCV, THI-Labeling-Tool, Google Colab, Linux 

    \resumeProjectHeading
      {\textbf{Optimierung des Kühlkörpers für Nvidia Jetson Nano}}{Okt 2022 -- Mär 2023}
      \resumeItemListStart
        \resumeItem{Entwicklung eines rechteckigen Kühlkörpers zur Senkung der Temperatur von 105°C auf 85°C.}
        \resumeItem{Durchführung einer Optimierungsstudie zur Bestimmung des effizientesten Designs, was zu einer Verbesserung der Kühlleistung um 20\% führte.}
        \resumeItem{Durchführung von Mesh-Konvergenzstudien zusammen mit parametrischen Analysen zur weiteren Verfeinerung des Designs.}
        \resumeItem{Zusammenarbeit mit einem Team zur Dokumentation und Präsentation der Ergebnisse in einem detaillierten technischen Bericht.}
      \resumeItemListEnd
      \hspace*{\labelsep}\textbf{Technologie-Stack:} Floefd 2021, MS Word
  \resumeSubHeadingListEnd

% Kurse
\section{Kurse}
  \resumeSubHeadingListStart
    \resumeSubheading
      {Full Stack Data Science und Künstliche Intelligenz}{Mai 2023 -- Sep 2023}
      {Naresh i Technologies}{Hyderabad, Indien}
      
    \resumeSubheading
      {Komplette Generative KI mit Langchain, Huggingface (NLP, LLM, GenAI)}{Jul 2024 -- heute}
      {Udemy}{}
  \resumeSubHeadingListEnd

% Fähigkeiten
\section{Fähigkeiten}
  \resumeSubHeadingListStart
    \resumeItem{\textbf{Programmierung:} Python (SciKit-Learn, Pandas, NumPy, OpenCV, Matplotlib), ROS (Robot Operating System)}
    \resumeItem{\textbf{Maschinenlernen \& Datenwissenschaft:} Deep Learning: PyTorch, TensorFlow, Keras; Datenvisualisierung: Matplotlib, Seaborn; Statistische Analyse und Datenmining: Scikit-learn}
    \resumeItem{\textbf{Software \& Tools:} MS Office, Matlab - Simulink, SolidWorks, Visual Studio, Git, Carla}
    \resumeItem{\textbf{Methoden:} Atlassian-Toolchain (Jira, Confluence, Bitbucket)}
    \resumeItem{\textbf{Betriebssysteme:} Windows, Linux}
    \resumeItem{\textbf{Soziale Fähigkeiten:} Teamarbeit, effektive Kommunikation, Problemlösung, analytisch, selbstständiges Arbeiten}
    \resumeItem{\textbf{Führerschein:} Klasse B}
  \resumeSubHeadingListEnd

% Sprachen
\section{Sprachen}
  \resumeSubHeadingListStart
    \resumeItem{\textbf{Englisch:} Fließende Arbeitskenntnisse}
    \resumeItem{\textbf{Deutsch:} Grundkenntnisse (B1)}
  \resumeSubHeadingListEnd

\end{document}
